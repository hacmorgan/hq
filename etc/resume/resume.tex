\documentclass[12pt]{article}
\usepackage[left=3cm, right=3cm, top=2.5cm]{geometry}  % set margins
\usepackage[document]{ragged2e}
\usepackage[utf8]{inputenc}
\usepackage[english]{babel}
\usepackage{graphicx}
\usepackage{hyperref}
\hypersetup{
  colorlinks=true,
  linkcolor=blue,
  filecolor=magenta,
  urlcolor=cyan,
  pdftitle={Hamish Morgan - Resume},
  pdfpagemode=FullScreen,
}
\setcounter{secnumdepth}{4}

%% \renewcommand{\familydefault}{\sfdefault}  % set font to this nice smooth one
%% \renewcommand{\familydefault}{\ttdefault}  % set font to this nice smooth one
%% \renewcommand{\baselinestretch}{1.5}

\begin{document}

\begin{center}
  \Large
  \vspace{0.8cm}
  \textbf{Hamish Morgan}\\
  \vspace{0.8cm}
  \large
  \href{https://github.com/hacmorgan}{GitHub @hacmorgan} - ham430@gmail.com - 0407245066

\end{center}


\begin{FlushLeft}

  \section{Experience}

  \subsection{Abyss Solutions (Sep 2018 - Present)}

  \subsubsection{Data Pipeline Tech Lead (Nov 2023 - Present)}
  Lead the data pipeline team, including implementing sweeping backend refactors, as well as general team leadership \\

  \begin{itemize}
  \item \textbf{General team lead stuff}, including scrum-mastering, presenting demos, leading retrospectives, mentoring junior/intermediates, etc. \\
  \item \textbf{Implement near-full backend rewrite}, consolidating a patchwork of inconsistently-styled scripts and config files into a unified workflow with consistent code style and documentation, built on top of the Prefect orchestration framework \\
  \item \textbf{Design of performant algorithm for highlighting elements in a scene based on spatial data}, like a heatmap, used to improve ML labelling accuracy \\
  \end{itemize}

  \subsubsection{Data Pipeline Software Engineer (Jul 2023 - Nov 2023)}
  Designed and implemented core data pipeline algorithms, workflows, and tools \\

  \begin{itemize}
  \item \textbf{Advanced usage of numpy and high-performance Python libraries} to implement data pipeline tooling \\
  \item \textbf{Replacing legacy Bash/C++ code with Python} to enable maintainability \\
  \item \textbf{Maintenance of core software library} including structural changes, to reduce complexity and enable rapid development \\
  \end{itemize}

  \subsubsection{Machine Learning Software Engineer (Oct 2021 - Jul 2023)}
  Designed and implemented advanced perception algorithms, and maintained the ML teams' tools and repository. \\

  \begin{itemize}
  \item \textbf{Advanced usage of Tensorflow ML Python library and strong knowledge of CUDA}. Used custom training loops to allow for complex model architectures, and custom logging for better insights into model performance during training. Also maintained tools for compiling custom Tensorflow layers over multiple versions of Tensorflow and CUDA. \\
  \item \textbf{Design and maintenance of performant high level tools for ML team}. Created easy to use visualisation tools for engineers and automated pipelines, developed high level (e.g. instancewise) metrics for ML systems, as well as various performant IO and data wrangling tools. \\
  \item \textbf{Design and implementation of MLOPS \& continuous learning systems}. Co-designed and developed a continuous learning system to pull new data and retrain an anomaly detection model, with human input only required to verify the model's performance and to approve moving it to production, assuming it improved.
  \item \textbf{Implementation of CI/CD pipelines, unit tests, etc. for ML team repository}. Maintined CI pipelines (using SemaphoreCI) running linters and unit tests to enforce code standards, and CD pipelines (also using SemaphoreCI) building docker images with code ready to run, in ML team and common software team repositories.
  \end{itemize}

  \subsubsection{Robotics Software Engineer (Jun 2019 - Oct 2021)}
  Designed, maintained, and executed bulk post-processing pipelines for data collected by robotics team.
  \begin{itemize}
  \item \textbf{Advanced use of photogrammetry tool Python API for complex, dimensionally accurate reconstructions}, allowing a highly complex workflow to scale with increasingly large and varied projects.
  \item \textbf{Design and maintenance of high level tools for photogrammetry pipelines}, allowing common workflows (e.g. camera calibration) to be done by robotics engineers alone.
  \item \textbf{Design of other data processing, visualisation and validation tools, and other robotics team enablers}, including sonar and lidar reconstruction tools, video annotation tools, in-field validation tools, etc. Additionally, maintained a system-wide installation of processing tools on robotics team compute server.
  \end{itemize}

  \subsubsection{Robotics Engineer (Sep 2018 - Jun 2019)}
  Working on hardware design and fabrication.
  \begin{itemize}
  \item \textbf{Component design, fabrication, and testing for robotic platforms}, e.g. designed components with CAD software (OnShape) and 3D printed them or fabricated with aluminium extrusion and other commonly available materials. Also designed and fabricated testing rigs for camera and lighting systems for data collection, then carried out tests, compiling and reporting results to robotics team.
  \item \textbf{Electrical and control system prototyping and fabrication}, e.g. populating circuitboards, driving thrusters, general arduino use. Designed and fabricated an autofocus system for a stereo machine vision camera system using an arduino and 3D printed parts.
  \item \textbf{Some field experience} - attended an inspection at Adelaide Dam.
  \end{itemize}

  \subsection{Beacon Lighting (Jan 2016 - Apr 2019)}
  Retail sales associate position

  \subsection{Dickson's Music (Nov 2013 - Aug 2014)}
  Retail sales associate position

  \section{Education}
  \begin{itemize}
  \item Bachelor of Engineering (Hons) in Mechatronic Engineering - UNSW. Honours thesis project detected vegetation in need of water/nutrients from a moving platform using basic computer vision methods.
  \item Certificate 2 in Video Game Programming - AIE
  \item Higher School Certificate - Shore School
  \end{itemize}

\end{FlushLeft}


\end{document}
